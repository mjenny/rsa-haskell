\documentclass[a4paper, 11pt]{article} % ISO-8859-1 - latin1

%Math
\usepackage{amsmath}
\usepackage{amsfonts}
\usepackage{amssymb}
\usepackage{amsthm}
\usepackage{ulem}
\usepackage{stmaryrd} %f\UTF{00FC}r Blitz!
\usepackage{tikz}

%PageStyle
\usepackage[ngerman]{babel} % deutsche Silbentrennung
\usepackage[utf8]{inputenc} % wegen deutschen Umlauten
\usepackage{fontenc}
\usepackage{fancyhdr, graphicx} %for header/footer
\usepackage{wasysym}
\usepackage{fullpage}
\usepackage{textcomp}

% Listings
\usepackage{color}
\usepackage{xcolor}
\usepackage{listings}
\usepackage{caption}

% Commands
\newcommand{\Bold}[1]{\textbf{#1}} %Boldface
\newcommand{\Kursiv}[1]{\textit{#1}} %Italic
\newcommand{\T}[1]{\text{#1}} %Textmode
\newcommand{\Nicht}[1]{\T{\sout{$ #1 $}}} %Streicht Shit durch
\newcommand{\lra}{\leftrightarrow} %Arrows
\newcommand{\ra}{\rightarrow}
\newcommand{\la}{\leftarrow}
\newcommand{\lral}{\longleftrightarrow}
\newcommand{\ral}{\longrightarrow}
\newcommand{\lal}{\longleftarrow}
\newcommand{\Lra}{\Leftrightarrow}
\newcommand{\Ra}{\Rightarrow}
\newcommand{\La}{\Leftarrow}
\newcommand{\Lral}{\Longleftrightarrow}
\newcommand{\Ral}{\Longrightarrow}
\newcommand{\Lal}{\Longleftarrow}
\newcommand{\RN}{\mathbb{R}} %Real Number
\newcommand{\NN}{\mathbb{N}} %Natural Number
\newcommand{\QN}{\mathbb{Q}} %Rational Number
\newcommand{\ZN}{\mathbb{Z}} %ganze Zahlen
\newcommand{\CN}{\mathbb{C}}

% Code listenings
\DeclareCaptionFont{white}{\color{white}}
\DeclareCaptionFormat{listing}{\colorbox{gray}{\parbox{\textwidth}{#1#2#3}}}
\captionsetup[lstlisting]{format=listing,labelfont=white,textfont=white}
 
\lstdefinestyle{JavaStyle}{
 language=Java,
 basicstyle=\footnotesize\ttfamily, % Standardschrift
 numbers=left,               % Ort der Zeilennummern
 numberstyle=\tiny,          % Stil der Zeilennummern
 stepnumber=1,              % Abstand zwischen den Zeilennummern
 numbersep=5pt,              % Abstand der Nummern zum Text
 tabsize=2,                  % Groesse von Tabs
 extendedchars=true,         %
 breaklines=true,            % Zeilen werden Umgebrochen
 frame=b,         
 %commentstyle=\itshape\color{LightLime}, Was isch das? O_o
 %keywordstyle=\bfseries\color{DarkPurple}, und das O_o
 basicstyle=\footnotesize\ttfamily,
 stringstyle=\color[RGB]{42,0,255}\ttfamily, % Farbe der String
 keywordstyle=\color[RGB]{127,0,85}\ttfamily, % Farbe der Keywords
 commentstyle=\color[RGB]{63,127,95}\ttfamily, % Farbe des Kommentars
 showspaces=false,           % Leerzeichen anzeigen ?
 showtabs=false,             % Tabs anzeigen ?
 xleftmargin=17pt,
 framexleftmargin=17pt,
 framexrightmargin=5pt,
 framexbottommargin=4pt,
 showstringspaces=false      % Leerzeichen in Strings anzeigen ?        
}

%Config
\renewcommand{\headrulewidth}{0pt}
\setlength{\headheight}{15.2pt}
\pagestyle{plain}

%Metadata
\title{Implementation des RSA Verschlüsselungs-Algorithmus in Haskell}
\author{Manuel Jenny \& Christian Glatthard, Projektteam 5}
\date{4. Semester (FS 2013)}


% hier beginnt das Dokument
\begin{document}

% Titelbild
\maketitle

\setcounter{page}{1}
\pagenumbering{arabic}

% Inhalt Start

\section{Abstract}
Dies ist ein Semesterprojekt im Modul Konzepte von Programmiersprachen an der Fachhochschule Nordwestschweiz. Ziel der Arbeit besteht darin Übung im Umgang mit funktionalen Programmiersprachen, insbesondere Haskell, zu bekommen.

\section{Idee des Projektes}
Ziel unseres Projektes ist es eine funktionierende Implementierung des Verschlüsselungssystem RSA umzusetzen. Diese wird nicht den Sicherheitsstandards für produktive RSA Verschlüsselungen entsprechen, sondern sie soll einen Überblick über die Funktionsweise des RSA Verschlüsselungsverfahrens bieten. Es werden daher sämtliche Funktionen bereitgestellen wie das Generieren von Keys (inkl. Primzahlenerkennung) sowie Ver- und Entschlüsselung von Strings.

Soweit möglich versuchen wir dabei die im Modul Kryptografie kennengelernten Algorithmen zu verwenden.

\section{Theoretischer Teil}
RSA (benannt nach den Erfindern Ron Rivest, Adi Shamir und Leonard Adleman) ist ein asymmetrisches kryptografisches Verschlüsselungsverfahren, welches sowohl zur Verschlüsselung, als auch zur digitalen Signatur verwendet werden kann.

Es wird ein privater und ein öffentlicher Schlüssel generiert. Der Öffentliche wird zum Verschlüsseln und zum Prüfen von Signaturen verwendet und ist öffentlich zugänglich. Der private Schlüssel hingegen wird zum Entschlüsseln, sowie zum Signieren der Daten verwendet und muss geheim bleiben.

In der Praxis wird RSA meist nur noch verwendet, um die Schlüssel eines anderen Verschlüsselungsverfahren zu verschlüsseln und diesen so sicher übertragen zu können. Für unser Projekt verwenden wir RSA jedoch direkt um Strings zu verschlüsseln.

\subsection{Vorgehen RSA}
\begin{enumerate}
\item Wähle 2 Primzahlen p, q
\item n = p*q
\item Wähle natürliche Zahl e, teilerfremd zu $\phi (n)$, d.h. $gcd(e,\phi(n)) = 1$, für die gilt $1 < e < \phi(n)$
\item Bestimme natürliche Zahl d mit $d=e^{-1}$, d.h. $e*d \equiv 1$ $mod$ $\phi (n) \Longleftrightarrow e*d*+k*\phi (n) = 1$
\end{enumerate}

\subsection{Verwendete mathematische Formeln und Algorithmen}
Für die Umsetzung von RSA sind einige mathematische Formeln und Definitionen unentbehrlich, diese haben wir hier aufgelistet.
\subsubsection{erweiterter euklidischer Algorithmus}
$ggT(a,b)= ax+by$
\subsubsection{modular multiplikatives Inverse}
$a^{-1}\equiv x $ $(mod$ $ m)$
\subsubsection{Eulersche Phi-Funktion}
Die eulersche Phi-Funktion gibt aus wieviele Zahlen $<$ n teilerfremd sind zu n.
$\phi (n) = { a \in N | 1 \leq a \leq n \and gcd(a,n) = 1 }$\linebreak
$\phi (mn) = \phi (m) * \phi (n)$
\subsubsection{Modulares Multiplizieren}
Um effizient mit grossen Exponenten zu arbeiten, benutzen wir die binäre Exponentiation in Kombination mit Modulo. Dieser Algorithmus ist wie folgt definiert:\\
Seien $a,b,n \in \ZN$ und $b,n > 1$. Berechne $a^b$ mod $n$.
\begin{itemize}
	\item[1.)] binäre Darstellung von b: $b=\sum_{i=0}^k \alpha_i2^i$ mit $\alpha \in \{0,1\}$.
	\item[2.)] Anwendung auf a: \\
		$a^{\sum_{i=0}^k \alpha_i2^i}$ = $\prod_{i=0}^{k} a^{\alpha_i2^i}$ = $a^{\alpha_k2^k}*a^{\alpha_{k-1}2^{k-1}}*a^{\alpha_{k-2}2^{k-2}} \dots  a^{\alpha_12}*a^{\alpha_0}$ = $($\dots$((a^{a_k})^2*a^{a_{k-1}})^2$\dots$*a^{\alpha_1})^2*a^{\alpha_0}$
	\item[3.)] Das Verfahren besteht nun darin, den letzten Ausdruck von innen nach aussen auszuwerten und nach jeder Multiplikation das Resultat modulo $n$ zu rechnen.
\end{itemize}
\section{Haskell-Code}
\subsection{Header}
Wir schreiben sämtlichen Programmcode als Modul, so dass dieser auch in anderen Projekten genutzt werden kann. Dazu müssen sowohl Name des Moduls, als auch die nach aussen angebotenen Funktionen definiert werden (generateKeyPair, encrypt, decrypt). Per Import werden weitere Module geladen, die unser Programm benötigt.
\lstinputlisting[language=haskell,caption=Modul Header,style=JavaStyle]{includes/module_header.hs}

\subsection{Funktion generateKeyPair}
Die exportierten Funktionen sind interaktiv programmiert. Durch Nutzung von System.IO können die Primzahlen sowie der öffentliche Schlüssel eingegeben werden. Der private Schlüssel d wird mithilfe des inversen Modulo berechnet.
Die Schlüssel werden je in einer separaten Datei gespeichert (pub.key, priv.key), und können so den entsprechenden Parteien zur Verfügung gestellt werden.
\lstinputlisting[language=haskell,caption=generateKeyPair,style=JavaStyle]{includes/generateKeyPair.hs}

\lstinputlisting[language=haskell,caption=enterPrimes,style=JavaStyle]{includes/enterPrimes.hs}

\subsection{Funktion isPrime}
Der folgende Code überprüft ob eine Zahl x eine Primzahl ist. Um die Performance zu steigern, wird vor der Listenabfrage überprüft, ob die Zahl x durch 2 (even), 3, 5 oder 7 teilbar ist. Ist dies der Fall, wird sofort False zurückgegeben.
Ist dies nicht der Fall, wird über die Funktion getDivisorList eine Liste aller Zahlen von 2 bis $\sqrt{x}$ generiert.

Diese Liste wird nun mit der Funktion isNotDivisor auf Teiler von x überprüft. Sobald ein Teiler gefunden wird, liefert sie False zurück. Wird kein Teiler gefunden, ist die Zahl eine Primzahl und die Funktion gibt True zurück.
\lstinputlisting[language=haskell,caption=Überprüfen ob Zahl Primzahl ist,style=JavaStyle]{includes/isPrime.hs}

\subsection{Funktion inverseMod}
Das modulare multiplikative Inverse Modulo benötigt, um den geheimen Teil des privaten Schlüssels zu berechnen. Dies geschieht mit Hilfe des erweiterten euklidischen Algorithmus. Die zugrundeliegenden mathematischen Formeln haben wir von Wikipedia.
\lstinputlisting[language=haskell,caption=inverseMod,style=JavaStyle]{includes/inversemod.hs}

\subsection{Funktion powerMod (Square \& Multiply)}
Es wird eine Hilfsfunktion, powerModExec definiert, welche die eigentliche Berechung übernimmt. Es wird die Liste von 0 und 1 abgearbeitet und das Resultat c zurückgegeben, sobald die Liste leer ist. Die Liste repräsentiert den Exponenten in binärer Form. Ist das erste Element der Liste eine 0 wird c (vorheriges Resultat) quadriert, modulo n genommen und als neues c rekursiv übergeben. Ist das erste Element der Liste eine 1 wird zusätzlich danach zusätzlich mit der Basis b multipliziert und das Resultat wieder modulo n genommen. Den so erhaltenen Rest wird dann als neues c übergeben.

\lstinputlisting[language=haskell,caption=powermod,style=JavaStyle]{includes/powermod.hs}

\subsection{Funktion encrypt}
Um einen String zu verschlüsseln wird der öffentliche Schlüssel benötigt. Dieser wird entweder aus einer zuvor mit generateKeyPair generierten oder einer von Hand erstellten Datei eingelesen.
Die Funktion nimmt einen String auf, teilt ihn in möglichst grosse Blöcke (Blocklänge $<$ n) und verschlüsselt diese daraufhin Zeichen für Zeichen. Als Zahlenwert für die Zeichen wird die UTF8 Codierung verwendet.
Damit bei der Entschlüsselung wieder dieselben Blöcke bearbeitet werden, wird der Cipher-Text (Geheimtext) in Form einer Integer-Liste ausgegeben, wobei jedes Element der Liste einen Nachrichten-Block representiert.
Dadurch ist natürlich die Sicherheit der Verschlüsselung bei kleinen Schlüsseln nicht gewährleistet, da gleiche Blöcke auch immer den gleichen Geheimtext produzieren. Nimmt man jedoch grosse Schlüssel, in der Praxis mindestens $2^{512}$, so wird es nahezu unmöglich die Verschlüsselung zu knacken.

% ------------------------ Probleme mit Zeichencodierung ? ---------------%


\lstinputlisting[language=haskell,caption=encrypt,style=JavaStyle]{includes/encrypt.hs}
\subsubsection{Funktion encryptString}
\lstinputlisting[language=haskell,caption=encryptString,style=JavaStyle]{includes/encryptString.hs}


\subsection{Funktion decrypt}
Für die Entschlüsselung wird eine Datei mit einem gültigen Privatekey benötigt. Der Geheimtext muss in Form einer Integer-Liste angegeben werden, so wie er bei der encrypt-Methode generiert wird.
Für die Entschlüsselung werden die Blöcke zuerst separat entschlüsselt und danach die einzelnen Zeichen wieder in UTF8 Zeichen umgewandelt und zu einem String zusammengesetzt.
\lstinputlisting[language=haskell,caption=decrypt,style=JavaStyle]{includes/decrypt.hs}
\subsubsection{Funktion decryptString}
\lstinputlisting[language=haskell,caption=decryptString,style=JavaStyle]{includes/decryptString.hs}

\section{Limitierungen}
Da dieses Projekt dazu dient Erfahrungen in der Programmierung mit Haskell zu sammeln, ist die RSA-Implementierung sehr einfach gehalten und sollte \textbf{NICHT} als Produktivsystem verwendet werden. Zudem müssen folgende Punkte bei der Angabe von Primzahlen berücksichtigt werden:
\begin{enumerate}
\item Die Primzahlen sollte nicht zu gross gewählt werden, da es bei zu grossen Zahlen ($p*q$ $=$ $n$) zu einem Overflow kommen kann ($n$ $<$ $max(Int32)$).
\item Die Rückführung der Integer-Liste in Cipher-Text wurde nicht berücksichtigt, da es hier zu Problemen mit der Zeichenkodierung kommen kann.
\end{enumerate}

\section{Testfälle}
\subsection{powerMod}
powerMod ist kritisch für das Verschlüsseln mit grossen Zahlen. Um sicher zu gehen, dass unsere Implementation korrekt ist, haben wir eine Testfunktion dafür geschrieben. Es ist zu beachten, dass dieser Test bei grossen Exponenten länger dauert!
\lstinputlisting[language=haskell,caption=testPowerMod,style=JavaStyle]{includes/testPowerMod.hs}

\subsection{encryptExec, decryptExec}
Um zu testen, dass die verschlüsselte Nachricht entschlüsselt wieder dieselbe ist, existiert auch hier eine Testfunktion:
\lstinputlisting[language=haskell,caption=testEncryptExec/testDecryptExec,style=JavaStyle]{includes/testEncryptDecryptExec.hs}
\textbf{Beispiel:} $e=65537$, $n=1098922499$, $m=123123$, $c=942137302$\\
testEncryptExec: $testEncryptExec$ $65537$ $1098922499$ $123123$ $942137302$\\
Liefert: $True$\\
testDecryptExec: $testDecryptExec$ $55783673$ $1098922499$ $942137302$ $123123$\\
Liefert: $True$

- encryptString
- decryptString

\section{Ehrlichkeitserklärung}
Hiermit erklären wir, dass wir die vorliegende Arbeit selbstständig und nur unter Zuhilfenahme der in den Kommentaren genannten Quellen angefertigt haben. Wir versichern zudem, diese Arbeit nicht bereits anderweitig als Leistungsnachweis verwendet zu haben.\\\\
Windisch, 4. Juni 2013


\end{document} % Inhalt Ende 
