\documentclass[a4paper, 11pt]{article} % ISO-8859-1 - latin1

%Math
\usepackage{amsmath}
\usepackage{amsfonts}
\usepackage{amssymb}
\usepackage{amsthm}
\usepackage{ulem}
\usepackage{stmaryrd} %f\UTF{00FC}r Blitz!

%PageStyle
\usepackage[ngerman]{babel} % deutsche Silbentrennung
\usepackage[utf8]{inputenc} % wegen deutschen Umlauten
\usepackage{fontenc}
\usepackage{fancyhdr, graphicx} %for header/footer
\usepackage{wasysym}
\usepackage{fullpage}
\usepackage{textcomp}

% Listings
\usepackage{color}
\usepackage{xcolor}
\usepackage{listings}
\usepackage{caption}

% Commands
\newcommand{\Bold}[1]{\textbf{#1}} %Boldface
\newcommand{\Kursiv}[1]{\textit{#1}} %Italic
\newcommand{\T}[1]{\text{#1}} %Textmode
\newcommand{\Nicht}[1]{\T{\sout{$ #1 $}}} %Streicht Shit durch
\newcommand{\lra}{\leftrightarrow} %Arrows
\newcommand{\ra}{\rightarrow}
\newcommand{\la}{\leftarrow}
\newcommand{\lral}{\longleftrightarrow}
\newcommand{\ral}{\longrightarrow}
\newcommand{\lal}{\longleftarrow}
\newcommand{\Lra}{\Leftrightarrow}
\newcommand{\Ra}{\Rightarrow}
\newcommand{\La}{\Leftarrow}
\newcommand{\Lral}{\Longleftrightarrow}
\newcommand{\Ral}{\Longrightarrow}
\newcommand{\Lal}{\Longleftarrow}

% Code listenings
\DeclareCaptionFont{white}{\color{white}}
\DeclareCaptionFormat{listing}{\colorbox{gray}{\parbox{\textwidth}{#1#2#3}}}
\captionsetup[lstlisting]{format=listing,labelfont=white,textfont=white}
 
\lstdefinestyle{JavaStyle}{
 language=Java,
 basicstyle=\footnotesize\ttfamily, % Standardschrift
 numbers=left,               % Ort der Zeilennummern
 numberstyle=\tiny,          % Stil der Zeilennummern
 stepnumber=1,              % Abstand zwischen den Zeilennummern
 numbersep=5pt,              % Abstand der Nummern zum Text
 tabsize=2,                  % Groesse von Tabs
 extendedchars=true,         %
 breaklines=true,            % Zeilen werden Umgebrochen
 frame=b,         
 %commentstyle=\itshape\color{LightLime}, Was isch das? O_o
 %keywordstyle=\bfseries\color{DarkPurple}, und das O_o
 basicstyle=\footnotesize\ttfamily,
 stringstyle=\color[RGB]{42,0,255}\ttfamily, % Farbe der String
 keywordstyle=\color[RGB]{127,0,85}\ttfamily, % Farbe der Keywords
 commentstyle=\color[RGB]{63,127,95}\ttfamily, % Farbe des Kommentars
 showspaces=false,           % Leerzeichen anzeigen ?
 showtabs=false,             % Tabs anzeigen ?
 xleftmargin=17pt,
 framexleftmargin=17pt,
 framexrightmargin=5pt,
 framexbottommargin=4pt,
 showstringspaces=false      % Leerzeichen in Strings anzeigen ?        
}

%Config
\renewcommand{\headrulewidth}{0pt}
\setlength{\headheight}{15.2pt}
\pagestyle{plain}

%Metadata
\title{kpsp Zwischenbericht}
\author{Manuel Jenny \& Christian Glatthard}
\date{4. Semester (FS 2013)}
% \fancyfoot[C]{If you use this documentation for a exam, you should offer a beer to the authors!}

% hier beginnt das Dokument
\begin{document}

% Titelbild
\maketitle
\thispagestyle{fancy}

\newpage

% Inhaltsverzeichnis
\pagenumbering{Roman}
\tableofcontents	  	


\newpage
\setcounter{page}{1}
\pagenumbering{arabic}

% Inhalt Start

\section{Abstract}


\section{Idee des Projektes}
Ziel unseres Projektes ist es eine funktionierende Implementation nach der Funktionsweise von RSA zu programmieren. Diese muss nicht unbedingt den Sicherheitsstandards des echten RSA entsprechen, soll aber sämtliche Funktionen bereitstellen wie das Generieren von Keys sowie Ver- und Entschlüsselung von Strings.
Soweit möglich versuchen wir dabei die im Modul Kryptografie kennengelernten Algorithmen zu verwenden.

\section{Theoretischer Teil}
RSA (benannt nach den Erfindern Ron Rivest, Adi Shamir, und Leonard Adleman) ist ein asymmetrisches kryptografisches Verschlüsselungsverfahren, welches sowohl zur Verschlüsselung, als auch zur digitalen Signatur verwendet werden kann.

Es wird ein privater und ein öffentlicher Schlüssel generiert. Der öffentliche wird zum Verschlüsseln und zum Prüfen von Signaturen verwendet und ist öffentlich zugänglich. Der private Schlüssel hingegen wird zum Entschlüsseln, sowie zum Signieren der Daten verwendet.

\subsection{Vorgehen RSA}
\begin{enumerate}
\item Wähle 2 Primzahlen p,q
\item n = p * q
\item Wähle natürliche Zahl e, teilerfremd zu $\phi (n)$, d.h. $gcd(e,\phi(n)) = 1$, für die gilt $1 < e < \phi(n)$
\item bestimme natürliche Zahl d mit $e*d*mod \phi (n) = 1$
\end{enumerate}

\subsection{Verwendete mathematische Formeln und Algorithmen}
\begin{itemize}
\item ggt (grösster gemeinsamer Teiler)
\item erweiterter Euklid
\item inverer Modulo
\item diverse Algorithmen zum Verifizieren von Primzahlen
\end{itemize}

\section{Haskell-Programme}
\lstinputlisting[language=haskell,caption=Modul Header,style=JavaStyle]{includes/module_header.hs}

\section{Testfälle}

\end{document} % Inhalt Ende 